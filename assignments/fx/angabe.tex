\documentclass[a4paper,10pt]{article}
\usepackage[utf8]{inputenc}
\usepackage{graphicx}
\usepackage{subfigure}
\usepackage{fancyhdr}
\usepackage{listings}
\usepackage{ gensymb }


%opening

\pagestyle{fancy} %eigener Seitenstil
\fancyhf{} %alle Kopf- und Fu�zeilenfelder bereinigen
\fancyhead[L]{SEP Abgabegespr�ch SS2015/16} %Kopfzeile links
\fancyhead[C]{} %zentrierte Kopfzeile
 %Kopfzeile rechts
\renewcommand{\headrulewidth}{0.4pt} %obere Trennlinie
\fancyfoot[C]{} %Seitennummer
\renewcommand{\footrulewidth}{0.4pt} %untere Trennlinie


\lstset{language=C++, basicstyle=\ttfamily\tiny}  

\begin{document}

\section*{FX-Market}

Entwickle eine einfache Software zur Umrechnung von Wechselkursen (Foreign Exchange).


\begin{description}
 \item[change $<$CCY1$>$ $<$CCY2$>$ $<$amount$>$] 
 Wechsle Geld in der angegebenen Menge.
 \item[show] Zeige das gesamte Verm�gen an.
\end{description}

Das Startguthaben soll mit 100 EUR angenommen werden.
Die Wechselkurse werden von einer Datei eingelesen.


Eine Beispiel Eingabedatei mit Wechselkursen \texttt{rates.txt}:
\begin{lstlisting}[frame=single]
EUR USD 1.13
EUR CHF 1.09
CHF USD 1.03
\end{lstlisting}
 

\subsection*{Beispiel}
\begin{verbatim}
> show 
EUR: 100.00
> change EUR USD 3
> show 
EUR: 97.00
USD: 3.39
> change EUR XXX 5
failed: unknown currency!
> change EUR CHF 10000
failed: not enough money!
> change EUR CHF -1
failed: negative amount!
\end{verbatim}

\subsection*{Aufruf}
\texttt{./fx}

\subsection*{Bewertung}
\begin{itemize}
 \item Dokumentation und Programmierstil: 1 Punkt
 \item Einhalten der OOP Konzepte: 1 Punkt
 \item Struktur \& Korrektheit: 1 Punkt
\end{itemize}

\newpage


\end{document}
