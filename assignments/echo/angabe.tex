\documentclass[a4paper,10pt]{article}
\usepackage[utf8]{inputenc}
\usepackage{graphicx}
\usepackage{subfigure}
\usepackage{fancyhdr}
\usepackage{listings}
\usepackage{ gensymb }


%opening

\pagestyle{fancy} %eigener Seitenstil
\fancyhf{} %alle Kopf- und Fu�zeilenfelder bereinigen
\fancyhead[L]{SEP Abgabegespr�ch SS2015/16} %Kopfzeile links
\fancyhead[C]{} %zentrierte Kopfzeile
 %Kopfzeile rechts
\renewcommand{\headrulewidth}{0.4pt} %obere Trennlinie
\fancyfoot[C]{} %Seitennummer
\renewcommand{\footrulewidth}{0.4pt} %untere Trennlinie


\lstset{language=C++, basicstyle=\ttfamily\tiny}  

\begin{document}

\section*{Echo}

Entwickle ein Programm welches einen echo-Befehl mit Platzhaltern unterst�zt.


\begin{description}
 \item[set $<$variable$>$ $<$value$>$] 
Setze die Variable auf den gew�nschten Wert
 \item[show] Liste alle Variablen und Werte auf.
  \item[echo $<$text$>$] Gib den angegebenen Text aus. Hier sind die Variablen einzusetzen. (siehe Beispiel)
\end{description}

 

\subsection*{Beispiel}
\begin{verbatim}
> set name Christoph
> set job Tutor
> set group 6
show
name: Christoph
job: Tutor
group: 6
> echo $name ist $job der Gruppe $group
Christoph ist Tutor der Gruppe 6
\end{verbatim}

\subsection*{Aufruf}
\texttt{./echo}

\subsection*{Bewertung}
\begin{itemize}
 \item Dokumentation und Programmierstil: 1 Punkt
 \item Einhalten der OOP Konzepte: 1 Punkt
 \item Struktur \& Korrektheit: 1 Punkt
\end{itemize}

\newpage


\end{document}
