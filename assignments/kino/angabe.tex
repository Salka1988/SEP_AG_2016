\documentclass[a4paper,10pt]{article}
\usepackage[utf8]{inputenc}
\usepackage{graphicx}
\usepackage{subfigure}
\usepackage{fancyhdr}
\usepackage{listings}
\usepackage{ gensymb }


%opening

\pagestyle{fancy} %eigener Seitenstil
\fancyhf{} %alle Kopf- und Fu�zeilenfelder bereinigen
\fancyhead[L]{SEP Abgabegespr�ch SS2015/16} %Kopfzeile links
\fancyhead[C]{} %zentrierte Kopfzeile
 %Kopfzeile rechts
\renewcommand{\headrulewidth}{0.4pt} %obere Trennlinie
\fancyfoot[C]{} %Seitennummer
\renewcommand{\footrulewidth}{0.4pt} %untere Trennlinie


\lstset{language=C++, basicstyle=\ttfamily\tiny}  

\begin{document}

\section*{Kino}

Entwickle eine einfache Software zur Reservierung von Kinositzpl�tzen.


\begin{description}
 \item[reserve $<$row$>$ $<$seat$>$] 
 Reserviere den gew�nschten Platz.
 \item[show] Zeige die Belegung des Saals an.
\end{description}

Der Kinosaal hat 6 Reihen zu je 5 Pl�tzen.
Es ist darauf zu achten, dass kein Platz mehrfach reserviert wird.
Weiters muss darauf geachtet werden, dass kein Platz ausserhalb reserviert werden kann.
 

\subsection*{Beispiel}
\begin{verbatim}
> show 
| | | | | |
| | | | | |
| | | | | |
| | | | | |
| | | | | |
| | | | | |
> reserve 1 2
ok!
> show 
| |x| | | |
| | | | | |
| | | | | |
| | | | | |
| | | | | |
| | | | | |
> reserve 1 2
ERR: occupied!
> reserve 10 5
ERR: out of bounds!
\end{verbatim}

\subsection*{Aufruf}
\texttt{./kino}

\subsection*{Bewertung}
\begin{itemize}
 \item Dokumentation und Programmierstil: 1 Punkt
 \item Einhalten der OOP Konzepte: 1 Punkt
 \item Struktur \& Korrektheit: 1 Punkt
\end{itemize}

\newpage


\end{document}
